%-----------------------------------------------------------------------------
%
%               Final write up for FPL 2011
%
% Name:         Sutton-Ripley-Final.tex
%
% Purpose:    Final write up for the lambda S final project
%
% Author:     Matthew Ripley
%					DJ Sutton
%
% Created:      December 13 2011
%
%-----------------------------------------------------------------------------


\documentclass[preprint]{sigplanconf}

% The following \documentclass options may be useful:
%
% 10pt          To set in 10-point type instead of 9-point.
% 11pt          To set in 11-point type instead of 9-point.
% authoryear    To obtain author/year citation style instead of numeric.

\usepackage{amsmath}

\begin{document}

\copyrightyear{2011} 
\copyrightdata{[to be supplied]} 

\title{Implementing Lambda C}
\subtitle{a subset of Lambda S}

\authorinfo{Matthew Ripley}
           {University of Colorado}
           {Matthew.Ripley@colorado.edu}
\authorinfo{DJ Sutton}
           {University of Colorado}
           {daniel.sutton@colorado.edu\begin{scriptsize}

\end{scriptsize}}

\maketitle

\begin{abstract}
This is the text of the abstract.
\end{abstract}

\category{CR-number}{subcategory}{third-level}

\terms
term1, term2

\keywords
keyword1, keyword2

\section{Introduction}

The text of the paper begins here.

\section{Related Work}

\section{Background}
text 
\\
$Expressions$ \cite{lambdas}
\begin{align}
E\ ::&= x\ |\ E\ E\ | \lambda\ x.E \\
&|\ \lbrace S\ in\ E \rbrace \\
&|\ Cond(E,E,E)\ |\ Pf_{k}(E_{1}\cdots E_{k}) \\
&|\ CN_{0}\ |\ CN_{k}(E_{1}\cdots E_{k}) 
\end{align}
\\
$Non Initial Expressions$
\begin{align}
E\ ::&= CN_{k}(SE_{1}\cdots SE_{k}) 
\end{align}
\\
$Statements$
\begin{align}
S\ ::&= \varepsilon\\
&|\ x = E\\
&|\ S;S 
\end{align}
\\
$alpha\ renaming$ 
\begin{align} 
\lambda x.e &\equiv \lambda t.(e[t/x]) \\
\lbrace x=e ; S\ in\ e_{0} \rbrace &\equiv \lbrace t = e; S\ in\ e_{0}  \rbrace [t/x]
\end{align}
\\
$Parallel\ Operator$
\begin{align}
\varepsilon ; S &\equiv S \\
S_{1} ; S_{2} &\equiv S_{2} ; S_{1}\\
S_{1} ; (S_{2} ; S_{3}) &\equiv (S_{1} ; S_{2}) ; S_{3}
\end{align}

\section{Lambda C Reduction Rules}

\section{Implementation}

\begin{verbatim}
(* variable type *)
type var  = string * int
\end{verbatim}

\begin{verbatim}
(* expression type *)
type exp =
  | Const of const
  | Var of var
  | Appl of exp * exp
  | Lambda of var * exp
  | Cond of exp * exp * exp
  | Letrec of stmt * exp
  | Pfk of opr * exp list
  | Cnk of builtIn * exp list 
\end{verbatim}

\subsection{$\alpha$ Renaming}

\subsection{Substitution}

\subsection{$\beta$-let reduction}

\subsection{Let Rec Reduction}
$(\lambda x.e_{1})e_{2} \rightarrow \lbrace t = e_{2}\ in\ e_{1}[t/x] \rbrace$

\section{Results and Conclusions}


% We recommend abbrvnat bibliography style.

\bibliographystyle{abbrvnat}
\bibliography{Sutton-Ripley-Final}

\end{document}
